% XeLaTeX can use any Mac OS X font. See the setromanfont command below.
% Input to XeLaTeX is full Unicode, so Unicode characters can be typed directly into the source.

% The next lines tell TeXShop to typeset with xelatex, and to open and save the source with Unicode encoding.

%!TEX TS-program = xelatex
%!TEX encoding = UTF-8 Unicode

\documentclass[12pt]{article}
\usepackage{geometry}                % See geometry.pdf to learn the layout options. There are lots.
\geometry{letterpaper}                   % ... or a4paper or a5paper or ... 
%\geometry{landscape}                % Activate for for rotated page geometry
%\usepackage[parfill]{parskip}    % Activate to begin paragraphs with an empty line rather than an indent
\usepackage{graphicx}
\usepackage{amssymb}
\usepackage{amsmath}

% Will Robertson's fontspec.sty can be used to simplify font choices.
% To experiment, open /Applications/Font Book to examine the fonts provided on Mac OS X,
% and change "Hoefler Text" to any of these choices.

\usepackage{fontspec,xltxtra,xunicode}
\defaultfontfeatures{Mapping=tex-text}
\setromanfont[Mapping=tex-text]{Hoefler Text}
\setsansfont[Scale=MatchLowercase,Mapping=tex-text]{Gill Sans}
\setmonofont[Scale=MatchLowercase]{Andale Mono}

\title{Computing the turbulent kinetic energy dissipation rate from lidar quasi-vertical
stares}
\author{Simon de Szoeke}
%\date{}                                           % Activate to display a given date or no date

\begin{document}
\maketitle

% For many users, the previous commands will be enough.
% If you want to directly input Unicode, add an Input Menu or Keyboard to the menu bar 
% using the International Panel in System Preferences.
% Unicode must be typeset using a font containing the appropriate characters.
% Remove the comment signs below for examples.

% \newfontfamily{\A}{Geeza Pro}
% \newfontfamily{\H}[Scale=0.9]{Lucida Grande}
% \newfontfamily{\J}[Scale=0.85]{Osaka}

% Here are some multilingual Unicode fonts: this is Arabic text: {\A السلام عليكم}, this is Hebrew: {\H שלום}, 
% and here's some Japanese: {\J 今日は}.

The stabilization table was rolling with the ship, so it sampled some
component of the mean horizontal wind.

%\hypertarget{the-ship-coordinate-system}

Define the ship coordinate to be ``northward'' toward the bow and
``eastward'' to starboard. The direction angle of the wind is from is 0° for wind
from the north, and 90° for wind from the east. The \(x\) coordinate points forward
(north), and the \(y\) coordinate points to
starboard (east). \(z\) is downward for a right-handed north, east, down (NED) coordinate system.

Pitch \(\theta\) (y-axis) and roll \(\phi\) (x axis) rotations are
defined in this coordinate system. Positive roll angle is a rotation of
the port upward; positive pitch angle is a rotation of the bow upward.

%\hypertarget{winds}

The vertical wind is \(w\). The ship-relative wind is \(W\).
Ship-relative horizontal wind is 
$ (U, V) = $ (-speed cos(dir), -speed
sin(dir)).
The Doppler lidar measures the radial velocity away from the lidar \(\hat{w}\) and 
\[
\hat{w}^2 = (-U \sin\theta)^2 
+ (V \cos \theta \sin \phi)^2 
+ (W \cos \theta \cos \phi)^2.
\]
We solve this for the air
velocity, \[
W^2 = \frac{ \hat{w}^2 
- (U \sin\theta)^2 
- (V \cos\theta \sin\phi)^2 }
{(\cos\theta \cos\phi)^2}
\].

%\hypertarget{vectors-among-the-lidar-and-two-target-volumes}

The radial vector to the target volume is \[
{\bf \hat{r}} =
\begin{pmatrix}
{\hat x} \\ 
{\hat y} \\ 
{\hat z}
\end{pmatrix}
=
{\rm range}
\begin{pmatrix}
-\sin\theta \\
 \cos\theta \sin\phi \\
 \cos\phi \cos\theta
\end{pmatrix}
\]
The displacement between two \emph{simultaneous} target volumes is \[
\bf{\hat r_1 - \hat r_2} = 
\begin{pmatrix}
{\hat x} \\
{\hat y} \\
{\hat z}
\end{pmatrix}
=
{\rm range}
\begin{pmatrix}
-\sin \theta_1-(-\sin \theta_2) \\
 \cos\theta_2 \sin \phi_1 - \cos\theta_2 \sin \phi_2 \\
 \cos\theta_1 \cos \phi_1 - \cos\theta_2 \cos\phi_2
\end{pmatrix}
\]

In the time \(\tau\) the ship moves relative to the air a horizontal
displacement of \((X, Y) = \tau(U,V)\). The total displacement of the
sample volumes is \[
{\bf r_2 - r_1} = 
\begin{pmatrix}
{\hat x} + X\\
{\hat y} + Y\\
{\hat z}
\end{pmatrix}
\] Expanding, the distance between the sample volumes is \[
r = |{\bf r_2 - r_1}| =
\left|
\begin{pmatrix}
\tau U - {\rm range}(\sin\theta_1 + \sin\theta_2) \\
\tau V + {\rm range}(\cos\theta_1 \sin\phi_1 - \cos\theta_2 \sin\phi_2) \\
  {\rm range}(\cos\theta_1 \cos\phi_1 - \cos\theta_2 \cos\phi_2)
\end{pmatrix}
\right|
\] This is the displacement argument of the (second order) structure
function \[
D(r) = \overline{(w_2-w_1)^2},
\] which behaves as \(D = N + Ar^{2/3}\). The structure function is
averaged as a function of \(r\). We will retain sample pairs of
\((D,r)\) and bin average in \(r\), perhaps with bins equally populated
by sample pairs.

%\hypertarget{ship-motion-fluctuations}

The ship heave, pitch, and roll also induce fluctuating velocities in
the transmitter that must be added. (The separation of scale between the
mean ship-relative velocity and these fast fluctuations breaks down in
maneuvers where the ship changes direction quickly.) Heave, pitch, and
roll is either measured at the lidar on the stabilization table, where
it can be added directly to the radial Doppler \(\hat w\), measured in
the ship orientation at the location of the lidar, or at a reference
location on the ship (see survey).

Presently, I read a Notre Dame VectorNav file for measuring and
correcting the motion of the table, but I don't know the quantities and
units. The ship's POSMV IMU is also available from a reference location.
If we use the POSMV, we need to calculate the velocities and moments
from the heave and angular rates.

Assuming the ship is rigid, the heave is the same everywhere. Angular
rates induce motion in proportion to the moment
\({\bf L} = (L_x, L_y, L_z)\) be the displacement vector from the
reference location (of the POSMV) to the lidar. At the lidar the ship
moves by \[
\dot{x} = -\dot\theta L_z\\
\dot{y} = \dot\phi L_z\\
\dot{z} = \dot\theta L_x - \dot\phi L_y.
\] 
Varying yaw is ignored.

\section{The structure function for arbitrary displacements}

We sample the vertical component of the Doppler (radial) velocity along
arbitrary displacements \(\mathbf{r}\), which may be either longitudinally oriented
along the resolved velocity component, or longitudinal to it. The Kolmogorov 
similarity for second order structure functions is in general
\[
D_{ij}(\mathbf{r}) = C_2 \epsilon^{2/3}r^{2/3} (4\delta_{ij} - r_ir_j/r^2)/3
\]
where \(C_2\) is a universal constant.
For sampling only the vertical coordinate \(i=j=1\),
\[
D_{11}(\mathbf{r}) = C_2 \epsilon^{2/3} r^{2/3} (4 - r_1^2/r^2)/3.
\]
where \(r_1\) is the vertical (longitudinal) displacement and \(r = |\mathbf{r}|\) is the scalar
distance between samples.

\end{document}  